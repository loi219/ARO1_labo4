\documentclass[a4paper]{article} %format de la feuille + type de document https://en.wikibooks.org/wiki/LaTeX/Document_Structure#Document_classes
%packages nécessaire pour nos besoins
\usepackage[utf8]{inputenc}
\usepackage[T1]{fontenc}
\usepackage[english,french]{babel}
\usepackage{amsmath}
\usepackage{amssymb,amsfonts,textcomp}
\usepackage{color}
\usepackage{array}
\usepackage{supertabular}
\usepackage{hhline}
\usepackage{hyperref}
\usepackage{capt-of}
\usepackage[pdftex]{graphicx}
\usepackage{sectsty}
\usepackage{tcolorbox}
\usepackage{textcomp}
\usepackage{courier}
\usepackage[font={small,it}]{caption}
\usepackage{float}
\usepackage{graphicx}
\usepackage{subcaption}
\usepackage{caption}

%Définition des couleurs
\definecolor{havelockBlue}{rgb}{0.004, 0.42, 0.73}
\definecolor{Monokaimagenta}{rgb}{0.86,0.08,0.24}

%utilisation de la couleur définie avant
%toutes les sections auront cette couleur
\sectionfont{\color{havelockBlue}}

%début du document
\begin{document}

%début d'un titre
\begin{titlepage}
            %centre les éléments
	\centering
	
	{\scshape\LARGE \color{Monokaimagenta} Laboratoire \\ Registres à décalage \par}
	
	%espace vertical de 1 mms
	\vspace{1cm}
	
	{\Large\itshape Johanna Melly \& Sven Rouvinez\par}
	
	%http://www.personal.ceu.hu/tex/spacebox.htm
	\vfill
	Professeur\par
	%met le texte en gras 
	\textbf{Carlos Andrés Pena} \par% ajoute une ligne 
	\vspace{1cm}
	Assistant\par
	\textbf{Gaëtan Matthey}
	
	\vfill

            %affiche la date actuelle
	{\large \today\par}
	
%fin de la page de titre
\end{titlepage}

%démarre un chapitre, les nombres se mettent automatiquement et seront incrémenté quand un autre \section est rencontré
%voir https://en.wikibooks.org/wiki/LaTeX/Document_Structure#Sectioning_commands
\section{Registre à décalage 4 bits}
Sur la base du registre à décalage vu dans le cours, concevoir et implémenter un registre à décalage de 4 bits qui peut décaler à gauche (SHL), décaler à droite (SHR), charger un nibble (LOAD) ou garder son contenu (HOLD). Le registre à décalage doit être réalisé à l’aide de flip-flops D et de multiplexeurs.\\%saut à la ligne


%début d'un encadré avec la couleur définie plus haut
\begin{tcolorbox}[colframe=Monokaimagenta,colback=white]
\paragraph{Registre à décalage 4 bits- Max 1 page } %démarre un paragraphe
Insérez une capture d’écran pour présenter votre bloc Registre à décalage 4 bits (Structure interne).
Accompagnez-le de commentaires et d’explications nécessaires à sa compréhension.
Remplacez le texte ci-dessus par vos réponses (à l’intérieur du cadre rouge)\\ 

%fin de l'encadré
\end{tcolorbox}


    
\end{document}
